\documentclass[letterpaper, 12pt]{article}

\textwidth=6.5in
\textheight=9.5in
\topmargin=-0.75in
\oddsidemargin=0.0in
\evensidemargin=0.0in

\usepackage{graphicx}
\usepackage{enumitem}
\usepackage{amssymb, amsmath}
\usepackage{xcolor}
\usepackage{listings}

\usepackage{times}
\usepackage{natbib}
\usepackage{etoolbox}
\usepackage{astjnlabbrev-jh}
\usepackage{bibentry}
\usepackage{ifthen}
\usepackage{epsfig}

\usepackage{commath}
\usepackage{rotating}

\usepackage{dirtree}
\usepackage{changepage}

% \lstset{
% language=Python,
% showstringspaces=false,
% formfeed=\newpage,
% tabsize=4,
% commentstyle=\itshape,
% basicstyle=\ttfamily,
% morekeywords={models, lambda, forms}
% }
 
% \newcommand{\code}[2]{
% \hrulefill
% \subsection*{#1}
% \lstinputlisting{#2}
% \vspace{2em}
% }

% Default fixed font does not support bold face
\DeclareFixedFont{\ttb}{T1}{txtt}{bx}{n}{10} % for bold
\DeclareFixedFont{\ttm}{T1}{txtt}{m}{n}{10}  % for normal
% twelve-sized ttm:
\DeclareFixedFont{\tttb}{T1}{txtt}{bx}{n}{12}  % for bold
\DeclareFixedFont{\tttm}{T1}{txtt}{m} {n}{12}  % for normal

\DeclareFixedFont{\ttnm}{T1}{txtt}{m}{n}{9.8}  % for normal

% Custom colors
\usepackage{color}
\definecolor{deepblue}  {rgb}{0.0, 0.0, 0.5}
\definecolor{deepred}   {rgb}{0.8, 0.0, 0.0}
\definecolor{deepgreen} {rgb}{0.0, 0.5, 0.0}
\definecolor{commentc}  {rgb}{0.5, 0.5, 0.5}
\definecolor{DodgerBlue}{rgb}{0.1, 0.6, 1.0}

% Python style for highlighting
\newcommand\pythonstyle{\lstset{
language=Python,
basicstyle = \ttm,
morekeywords = {self, as, assert, with, yield}, % Add keywords here
keywordstyle = \ttb\color{blue},       %
emph        = {MyClass, __init__},     % Custom highlighting
emphstyle   = \ttb\color{DodgerBlue},  % Custom  highlighting style
stringstyle = \color{deepred},         % Strings highlighting style
commentstyle=\color{commentc},         % Comment highlighting style
frame       = tb,                      % Any extra options here
showstringspaces = false
}}

% Python environment:
\lstnewenvironment{python}[1][]{\pythonstyle\lstset{#1}}{}
% Python for external files:
\newcommand\pythonexternal[2][]{{\pythonstyle\lstinputlisting[#1]{#2}}}
% Python for inline:
\newcommand\pythoninline[1]{{\pythonstyle\lstinline!#1!}}


% Python style for highlighting
\newcommand\plainstyle{\lstset{
language=Python,
basicstyle = \ttnm,
keywordstyle= \ttnm,                %
emph        = {MyClass, __init__},  % Custom highlighting
emphstyle   = \ttnm\color{black},   % Custom  highlighting style
stringstyle = \color{black},        % Strings highlighting style
commentstyle=\color{black},         % Comment highlighting style
frame       = tb,                   % Any extra options here
showstringspaces = false
}}

% Plain environment:
\lstnewenvironment{plain}[1][]{\plainstyle\lstset{#1}}{}

\newcommand\plaininline[1]{{\plainstyle\lstinline!#1!}}

% To use boldface verbatim:
%\lstset{basicstyle=\ttfamily,
%        escapeinside={||},
%        mathescape=true}

\lstset{
    language={[LaTeX]TeX},
    basicstyle=\tt\color{red},
    escapeinside={||},
}

\bibliographystyle{apj_hyperref}
\usepackage[%pdftex,      %%% hyper-references for pdflatex
bookmarks=true,           %%% generate bookmarks ...
bookmarksnumbered=true,   %%% ... with numbers
colorlinks=true,          % links are colored
citecolor=blue,           % green   % color of cite links
linkcolor=blue,           %cyan,         % color of hyperref links
menucolor=blue,           % color of Acrobat Reader menu buttons
urlcolor=blue,            % color of page of \url{...}
breaklinks=true,
linkbordercolor={0 0 1},  %%% blue frames around links
pdfborder={0 0 1},
frenchlinks=true]{hyperref}
%\usepackage{breakurl}

\newcommand{\eprint}[1]{\href{http://arxiv.org/abs/#1}{#1}}
\newcommand{\ISBN}[1]{\href{http://cosmologist.info/ISBN/#1}{ISBN: #1}}
\providecommand{\adsurl}[1]{\href{#1}{ADS}}

% hyper ref only the year in citations:
\makeatletter
% Patch case where name and year are separated by aysep:
\patchcmd{\NAT@citex}
  {\@citea\NAT@hyper@{%
     \NAT@nmfmt{\NAT@nm}%
     \hyper@natlinkbreak{\NAT@aysep\NAT@spacechar}{\@citeb\@extra@b@citeb}%
     \NAT@date}}
  {\@citea\NAT@nmfmt{\NAT@nm}%
   \NAT@aysep\NAT@spacechar\NAT@hyper@{\NAT@date}}{}{}
% Patch case where name and year are separated by opening bracket:
\patchcmd{\NAT@citex}
  {\@citea\NAT@hyper@{%
     \NAT@nmfmt{\NAT@nm}%
     \hyper@natlinkbreak{\NAT@spacechar\NAT@@open\if*#1*\else#1\NAT@spacechar\fi}%
       {\@citeb\@extra@b@citeb}%
     \NAT@date}}
  {\@citea\NAT@nmfmt{\NAT@nm}%
   \NAT@spacechar\NAT@@open\if*#1*\else#1\NAT@spacechar\fi\NAT@hyper@{\NAT@date}}
  {}{}
\makeatother


%\def\bibAnnoteFile#1{}
%\bibpunct[, ]{(}{)}{,}{a}{}{,}

% Packed reference list:
\setlength\bibsep{0pt}

% \pagestyle{myheadings}
% \markright{MC\sp{3}}
% \pagenumbering{arabic}


% :::::::::::::::::::::::
\newcommand\degree{\degr}
\newcommand\degrees{\degree}
\newcommand\vs{\emph{vs.}}

% unslanted mu, for ``micro'' abbrev.
\DeclareSymbolFont{UPM}{U}{eur}{m}{n}
\DeclareMathSymbol{\umu}{0}{UPM}{"16}
\let\oldumu=\umu
\renewcommand\umu{\ifmmode\oldumu\else\math{\oldumu}\fi}
\newcommand\micro{\umu}
\newcommand\micron{\micro m}
\newcommand\microns{\micron}

\let\oldsim=\sim
\renewcommand\sim{\ifmmode\oldsim\else\math{\oldsim}\fi}
\let\oldpm=\pm
\renewcommand\pm{\ifmmode\oldpm\else\math{\oldpm}\fi}
\newcommand\by{\ifmmode\times\else\math{\times}\fi}
\newcommand\ttt[1]{10\sp{#1}}
\newcommand\tttt[1]{\by\ttt{#1}}
\newcommand\tablebox[1]{\begin{tabular}[t]{@{}l@{}}#1\end{tabular}}
\newbox{\wdbox}
\renewcommand\c{\setbox\wdbox=\hbox{,}\hspace{\wd\wdbox}}
\renewcommand\i{\setbox\wdbox=\hbox{i}\hspace{\wd\wdbox}}
\newcommand\n{\hspace{0.5em}}
\newcommand\marnote[1]{\marginpar{\raggedright\tiny\ttfamily\baselineskip=9pt #1}}
\newcommand\herenote[1]{{\bfseries #1}\typeout{======================> note on page \arabic{page} <====================}}
\newcommand\fillin{\herenote{fill in}}
\newcommand\fillref{\herenote{ref}}
\newcommand\findme[1]{\herenote{(FINDME: #1)}}

\newcount\timect
\newcount\hourct
\newcount\minct
\newcommand\now{\timect=\time \divide\timect by 60
         \hourct=\timect \multiply\hourct by 60
         \minct=\time \advance\minct by -\hourct
         \number\timect:\ifnum \minct < 10 0\fi\number\minct}

\newcommand\citeauthyear[1]{\citeauthor{#1} \citeyear{#1}}

\newcommand\mc{\multicolumn}
\newcommand\mctc{\multicolumn{2}{c}}


% {\tttm -h, --help} \\
% Print the list of arguments. \newline

% \newenvironment{myindentpar}[1]%
%   {\begin{list}{}%
%          {\setlength{\leftmargin}{3cm}}%
%      \item[]%
%   }
% {\end{list}}

\newenvironment{packed_enum}{
\begin{enumerate}[leftmargin=3cm]
   \setlength{\itemsep}{1pt}
   \setlength{\parskip}{5pt}
   \setlength{\parsep}{0pt}
}{\end{enumerate}}

\newcommand{\argument}[2]{{\noindent\tttm #1}%
\begin{adjustwidth}{2.5em}{0pt}%
#2 \vspace{0.3cm}%
\end{adjustwidth}%
}

%\newcommand{\ttmb}[1]{\tttm\color{#1}}
\newcommand{\routine}[2]{{\noindent\tttm\color{blue} #1:}%
\begin{adjustwidth}{2.0em}{0pt}%
#2 \vspace{0.15cm}%
\end{adjustwidth}%
}

% \newcommand{\routine}[2]{{\noindent\tttm\color{blue} #1}%
% \begin{adjustwidth}{0.0em}{0pt}%
%  #2 \vspace{0.15cm}%
% \end{adjustwidth}%
% }

% :::::::::::::::::: jhmacs2.tex :::::::::::::::::::::::::::::::::::::
\typeout{Joe Harrington's personal setup, Wed Jun 17 10:53:17 EDT 1998}
% Tue Mar 29 22:23:03 EST 1994

% :::::: pato.tex ::::::
% Joetex character unreservations.
% This file frees most of TeX's reserved characters, and provides
% several alternatives for their functions.


% utility
\catcode`@=11

% comments are first....
\newcommand\comment[1]{}

\newcommand\commenton{\catcode`\%=14}
\newcommand\commentoff{\catcode`\%=12}

% Not-a-comment:
\newcommand\nocomment[1]{#1}

\renewcommand\math[1]{$#1$}
\newcommand\mathshifton{\catcode`\$=3}
\newcommand\mathshiftoff{\catcode`\$=12}

\comment{alignment tab}
\let\atab=&
\newcommand\atabon{\catcode`\&=4}
\newcommand\ataboff{\catcode`\&=12}

\let\oldmsp=\sp
\let\oldmsb=\sb
\def\sp#1{\ifmmode
           \oldmsp{#1}%
         \else\strut\raise.85ex\hbox{\scriptsize #1}\fi}
\def\sb#1{\ifmmode
           \oldmsb{#1}%
         \else\strut\raise-.54ex\hbox{\scriptsize #1}\fi}
\newbox\@sp
\newbox\@sb
\def\sbp#1#2{\ifmmode%
           \oldmsb{#1}\oldmsp{#2}%
         \else
           \setbox\@sb=\hbox{\sb{#1}}%
           \setbox\@sp=\hbox{\sp{#2}}%
           \rlap{\copy\@sb}\copy\@sp
           \ifdim \wd\@sb >\wd\@sp
             \hskip -\wd\@sp \hskip \wd\@sb
           \fi
        \fi}
\def\msp#1{\ifmmode
           \oldmsp{#1}
         \else \math{\oldmsp{#1}}\fi}
\def\msb#1{\ifmmode
           \oldmsb{#1}
         \else \math{\oldmsb{#1}}\fi}
\def\supon{\catcode`\^=7}
\def\supoff{\catcode`\^=12}
\def\subon{\catcode`\_=8}
\def\suboff{\catcode`\_=12}
\def\supsubon{\supon \subon}
\def\supsuboff{\supoff \suboff}


\newcommand\actcharon{\catcode`\~=13}
\newcommand\actcharoff{\catcode`\~=12}

\newcommand\paramon{\catcode`\#=6}
\newcommand\paramoff{\catcode`\#=12}

\comment{And now to turn us totally on and off...}

\newcommand\reservedcharson{ \commenton  \mathshifton  \atabon  \supsubon
                             \actcharon  \paramon}

\newcommand\reservedcharsoff{\commentoff \mathshiftoff \ataboff \supsuboff
                             \actcharoff \paramoff}

\newcommand\nojoe[1]{\reservedcharson #1 \reservedcharsoff}

\catcode`@=12
\reservedcharsoff

\reservedcharson
\newcommand\jhauth[1]{{#1}}
\newcommand\jhstud[1]{{#1}}

\comment{Must have ONLY ONE of these... trust these macros, they work
\newcommand\jhauth[1]{{\bfseries #1}}
\newcommand\jhstud[1]{{\em #1}}
}

\reservedcharsoff
\reservedcharson


% ::::::::::::::::::::::::::::::::::::::::::::::::::::

\def\vs{{\em vs.}}
\def\p{\phantom{(0)}}

% Section levels:
\setcounter{secnumdepth}{5}
%  \section{}       % level 1
%  \subsection{}    % level 2
%  \subsubsection{} % level 3
%  \paragraph{}     % level 4 - equivalent to subsubsubsection
%  \subparagraph{}  % level 5

% To show in the table of content:
\setcounter{tocdepth}{5}

% Linebreak after \paragraph
\makeatletter
\renewcommand\paragraph{%
   \@startsection{paragraph}{4}{0mm}%
      {-\baselineskip}%
      {.5\baselineskip}%
      {\normalfont\normalsize\bfseries}}
\makeatother

% Linebreak after \paragraph
\makeatletter
\renewcommand\subparagraph{%
   \@startsection{subparagraph}{4}{0mm}%
      {-\baselineskip}%
      {.5\baselineskip}%
      {\normalfont\normalsize\bfseries}}
\makeatother

\actcharon
\renewcommand{\textfraction}{0.1}
\comment{\paramon\def\herenote#1{}\paramoff}
\renewcommand{\thepage}{\arabic{page}}
\reservedcharson

% :::::::::::: My Additions ::::::::::::::
\newcommand\Spitzer{{\em Spitzer}}
\newcommand\SST{{\em Spitzer Space Telescope}}
\newcommand\chisq{$\chi^2$}
\newcommand\itbf[1]{\textit{\textbf{#1}}}
\newcommand\bftt[1]{\texttt{\textbf{#1}}}
\newcommand\function[1]{\noindent\texttt{\begin{tabular}{@{}l@{}l}#1\end{tabular}}\newline}
\newcommand\bfv[1]{|\textbf{#1}|}
\newcommand\ttred[1]{\textcolor{red}{\ttfamily #1}}
\newcommand\ttblue[1]{\textcolor{blue}{\ttfamily #1}}
\newcommand\ttblack[1]{\textcolor{black}{\ttfamily #1}}
\newcommand\der{{\rm d}}
\newcommand\tno{$\sp{-1}$}
\newcommand\tnt{$\sp{-2}$}
\newcommand*\Eval[3]{\left.#1\right\rvert_{#2}^{#3}}
\newcommand\mcc{MC\sp{3}}
\newcommand\transit{{\tt Transit}}
\newcommand\pylineread{{\tttm Pylineread}}
%:::::::::::::::::::::::::::::::::::::::::
% Next six lines adjust spacing above/below captions and Sections etc
% Adjust as needed

\comment{
% \setlength{\abovecaptionskip}{0pt}
% \setlength{\belowcaptionskip}{0pt}
% \setlength{\textfloatsep}{8pt}
% \titlespacing{\section}{0pt}{5pt}{*0}
% \titlespacing{\subsection}{0pt}{5pt}{*0}
% \titlespacing{\subsubsection}{0pt}{5pt}{*0}
}

\reservedcharsoff
\actcharon
\mathshifton

\reservedcharson


\begin{document}

\begin{titlepage}
\begin{center}

\textsc{\LARGE University of Central Florida}\\[1.5cm]

% Title
\rule{\linewidth}{0.5mm} \\[0.4cm]
{ \huge \bfseries BARTTest Users Manual \\[0.4cm] }
\rule{\linewidth}{0.5mm} \\[1.0cm]

\textsc{\Large A Test Suite for Radiative-Transfer and Retrieval Codes}\\[1.5cm]

% Author and supervisor
\noindent
\begin{minipage}{0.4\textwidth}
\begin{flushleft} \large
\emph{Author:}\\
Michael D. \textsc{Himes}\\
\end{flushleft}
\end{minipage}%
\begin{minipage}{0.4\textwidth}
\begin{flushright} \large
\emph{Supervisor:} \\
Dr.~Joseph \textsc{Harrington}
\end{flushright}
\end{minipage}

\vfill

% Bottom of the page
{\large \today}

\end{center}
\end{titlepage}

\tableofcontents

\newpage

\section{Team Members}
\label{sec:team}

\begin{itemize}
\item Michael D. Himes, University of Central Florida (mhimes@knights.ucf.edu).
\item Joseph Harrington, University of Central Florida (jh@physics.ucf.edu).
\item Ryan Challener, University of Central Florida (rchallen@knights.ucf.edu).
\item Patricio Cubillos, Space Research Institute, Austrian Academy of Sciences (patricio.cubillos@oeaw.ac.at)
\item Jasmina Blecic, New York University Abu Dhabi (jasmina@nyu.edu).
\item M. R. Green, University of Central Florida (mraechelgreen@knights.ucf.edu)
\end{itemize}

\section{Introduction}
\label{sec:intro}

This document describes BARTTest, a collection of tests for 
radiative-transfer (RT) and retrieval codes. 
The tests include simple forward models with 
analytically-verifiable answers, realistic forward models for an 
HD 189733b-like planet with various types of temperature profiles, and 
retrievals on synthetic and real data. The simple models test the 
presence, shape, and strength of lines and other basic aspects of RT codes. 
The realistic forward models and retrievals provide results that can be 
compared with other codes.

BARTTest is configured by default to test BART, the Bayesian Atmospheric 
Radiative Transfer code. The configuration and input files are provided 
as reference for users to see the exact setup for each test so that it 
can be reproduced by other codes. This manual provides a description 
of each test as well as each individual file for further clarity.

This manual describes the BARTTest program usage for the user
(Sections \ref{sec:setup} through \ref{sec:example}) and
potential contributors (Section \ref{sec:howto}).  Section \ref{sec:setup} 
indicates the system requirements and how to obtain the code. 
It also details the commands to run the program.
Section \ref{sec:files} describes the directories and files contained in 
BARTTest.  Section \ref{sec:example} shows how to execute BARTTest from start 
to finish for BART.  Section \ref{sec:howto} details how a user would 
contribute the results of their code to BARTTest. 

\newpage

\subsection{BARTTest Overview}

\noindent BARTTest is organized as follows:

\framebox{\begin{minipage}[t]{0.95\columnwidth}%
\dirtree{%
 .1 BARTTest. 
 .2 code-output. 
 .3 01BART. 
 .3 {\textless}other codes{\textgreater}. 
 .2 doc. 
 .2 lib. 
 .2 results. 
 .3 01BART. 
 .3 {\textless}other codes{\textgreater}. 
 .3 plots.
 .2 tests. 
 .3 00inputs.
 .4 filters.
 .4 par.
 .4 TLI.
 .3 c01hjcleariso.
 .3 c02hjclearnoinv.
 .3 c03hjclearinv.
 .3 f01oneline. 
 .3 f02fewline.
 .3 f03multiline.
 .3 f04broadening.
 .3 f05abundance.
 .4 atm.
 .3 f06blending.
 .3 f07multicia.
 .3 f08isothermal.
 .3 f09energycons.
 .3 r01hd189733b
 .3 s01hjcleariso.
 .3 s02hjclearnoinv.
 .3 s03hjclearinv.
}
\end{minipage}}
\vspace{0.7cm}

Each directory and the files within will be discussed in detail in 
Section \ref{sec:files}.

\pagebreak
\subsection{License}
\label{sec:license}

\section{Setup}
\label{sec:setup}

\subsection{System Requirements}
\label{sec:requirements}
BARTTest is configured to run on Linux; there are terminal commands executed 
that are Linux-specific, so users would have to modify this to run on other 
operating systems. The hardware requirements to run BARTTest are dependent on 
which tests the user wishes to execute. The isothermal (f08), energy conservation (f09), comparison (c01 - c03), and 
retrieval (r01, s01-s03) tests all have greater requirements for storage and RAM. The retrievals are by default set to use 10 CPU 
cores. To use less, the user must alter the configuration files; at least 3 cores are required to ensure 
proper sampling of the parameter space.

Requirements (excluding isothermal, comparison, and retrieval tests):
\begin{itemize} \itemsep0pt
  \item 1-core CPU
  \item 1 GB storage
  \item 6 GB RAM
\end{itemize}

Requirements (all tests):
\begin{itemize} \itemsep0pt
  \item 3-core CPU
  \item 40 GB storage
  \item 16 GB RAM
\end{itemize}

\subsection{Execution Methods}
\label{sec:commands}
The user has a number of ways to run the tests depending on where they are at 
in the process. The general process is as follows when running BARTTest for 
BART:
\begin{itemize} \itemsep0pt
  \item Pull BART from Github and compile
  \item Pull line lists (note: only necessary for 
        isothermal, energy conservation, comparison, and retrieval tests)
  \item Run the tests
  \item Produce plots of the results
\end{itemize}

The program is controlled by a Makefile in the top-level directory; all 
commands to execute BARTTest thus take the form of 
\texttt{make {\textless}command{\textgreater}}, executed 
from the top-level directory. The configured commands are listed in 
Table \ref{tbl:makecommands}.

\begin{table}[ht]
\label{tbl:makecommands}
\centering
\caption{Makefile Commands}
\begin{tabular}{ll}
\hline\hline
Command & Description\\
\hline
bart & Downloads BART from Github to BARTTest/../BART/ and compiles.\\
hitran{\_}linelists & Downloads necessary line lists from HITRAN for the 
isothermal, energy conservation, comparison, and retrieval tests.\\
oneline    & Executes the f01oneline test (see Section \ref{sec:oneline}).\\
fewline    & Executes the f02fewline test (see Section \ref{sec:fewline}).\\
multiline  & Executes the f03multiline test (see Section \ref{sec:multiline}).\\
broadening & Executes the f04broadening test (see Section \ref{sec:broadening}).\\
abundance  & Executes the f05abundance test (see Section \ref{sec:abundance}).\\
blending   & Executes the f06blending test (see Section \ref{sec:blending}).\\
multicia   & Executes the f07multicia test (see Section \ref{sec:multicia}).\\
isothermal & Executes the f08isothermal test (see Section \ref{sec:isothermal}).\\
energycons & Executes the f09energycons test (see Section \ref{sec:energycons}).\\
comparison{\_}tli & Generates the TLI file for the c01hjcleariso, c02hjclearnoinv, and c03hjclearinv tests (see Sections \ref{sec:comp-iso} -- \ref{sec:comp-inv}).\\
comparison{\_}iso & Runs the isothermal emission and transmission cases for the c01hjcleariso test. Note: This may take a couple hours.\\
comparison{\_}noinv & Runs the noninverted emission and transmission cases for the c02hjclearnoinv test. Note: This may take a couple hours.\\
comparison{\_}inv & Runs the inverted emission and transmission cases for the c03hjclearinv test. Note: This may take a couple hours.\\
comparison{\_}plots & Produces the plots that compare Transit's results to Ian Dobbs-Dixon's radiative hydrodynamic (RHD) code.\\
comparisontests & Generates TLI file, runs all of the cases for the c01hjcleariso, c02hjclearnoinv, and c03hjclearinv tests, and produces plots (see see Sections \ref{sec:comp-iso} -- \ref{sec:comp-inv}). Note: this takes on the order of hours.\\
synthretrievals & Runs all of the synthetic retrievals (s01 -- s03; see Section \ref{sec:retrievals}). Note: this may take on the order of a week.\\
retrieval{\_}tli & Generates the TLI file for the s01 -- s03 tests. Note: equivalent to comparison{\_}tli.\\
retrieval{\_}iso_e & Runs emission retrieval for the s01hjcleariso test. Note: this takes on the order of days.\\
retrieval{\_}iso_t & Runs transmission retrieval for the s01hjcleariso test. Note: this takes on the order of days.\\
retrieval{\_}noinv_e & Runs emission retrieval for the s02hjclearnoinv test. Note: this takes on the order of days.\\
retrieval{\_}noinv_t & Runs transmission retrieval for the s02hjclearnoinv test. Note: this takes on the order of days.\\
retrieval{\_}inv_e & Runs emission retrieval for the s03hjclearinv test. Note: this takes on the order of days.\\
retrieval{\_}inv_t & Runs transmission retrieval for the s03hjclearinv test. Note: this takes on the order of days.\\
hd189{\_}tli & Generates the TLI file for the HD 189733b retrieval (see Section \ref{sec:hd189}).\\
hd189{\_}retrieval & Runs a retrieval using HD 189733b secondary eclipse data. Note: this takes on the order of days (see Section \ref{sec:hd189}).\\
hd189 & Equivalent to executing hd189{\_}tli and hd189{\_}retrieval Note: this takes on the order of days (see Section \ref{sec:hd189}).\\
quicktests & Runs f01 -- f07.\\
forwardtests & Runs f01 -- f09.\\
all & Downloads BART and HITRAN/HITEMP line lists, and runs f01 -- f09 and c01 -- c03. Note: does not execute the retrievals due to the time involved.\\
plots & Produces plots of the code output for f01 -- f09, c01 -- c03, and s01 -- s03.\\
retrievalplots & Produces plots that compare the retrieved results to the known ground-truth for the synthetic retrieval cases.
clean & Deletes everything in BARTTest/code-output/01BART/ and BARTTest/results/01BART/.\\
\hline
\end{tabular}
\end{table}

These are the built-in methods of executing BARTTest. If the user wishes to 
run the package in some other manner, it must be manually configured.

\section{Descriptions of Directories and Files}
\label{sec:files}

This section describes the files and subdirectories in BARTTest.

\subsection{Top-level Files}
The BARTTest directory contains 3 files: LICENSE, Makefile, and README.

\begin{itemize} \itemsep0pt
  \item LICENSE: identical to Section \ref{sec:license}. It contains the 
        usage license.
  \item Makefile: the main driver of the program; it handles commands used 
        to execute tests, download line lists, and so on. See Section 
        \ref{sec:commands} for more information.
  \item README: a very brief version of this manual.
\end{itemize}

\subsection{code-output}
\label{sec:output}
The code-output directory contains subdirectories for the output of each RT/retrieval code. 
Each subdirectory is numbered according to the time that the output was received. 
For example, there is a directory ``01BART'' which contains BART's output for each of the tests as it was the first code to be tested.

\subsection{doc}
The doc directory contains documentation related to BARTTest, which you are currently reading.

\subsection{lib}
The lib directory contains various functions and scripts used in the execution 
of BARTTest.

Files in the directory:
\begin{itemize} \itemsep0pt
  \item abuncomp.py: contains a function to plot the results of the f05abundance test (see Section \ref{sec:abundance}). Produces the plot when executed.
  \item comparison.py: contains a function to plot the results of the c01 -- c03 tests for multiple 
        codes. When executed, it produces plots to compare BART and RHD.
  \item energycons.py: contains a function to process the results of the f09energycons test (see Section \ref{sec:energycons}). Produces a text file with the results when executed.
  \item makedepths.py: contains a function to take the outputs from the c01 -- c03 tests and produce the synthetic eclipse/transit depths for the s01 -- s03 tests. Note: uncertainties are exclusively due to photon noise.
  \item makeplots.py: contains functions used to plot the results of the f01 -- f09 tests, 
        and produces those plots when executed.
  \item opacityconv.py: contains a function to read a (binary) Transit opacity 
        file, convert it to Numpy arrays, and save out a .npz file.
  \item retrievalplots.py: contains functions to plot the results of the s01 -- s03 tests against the known ground-truth answers. Produces those plots when executed.
  \item voigtcomp.py: contains a function to compare the line profile from the 
        broadening test with the normalized theoretical Voigt profile 
        corresponding to the setup (temperature, pressure, molecular 
        properties, layer abundances). Used in the f04broadening test (see 
        Section \ref{sec:broadening}).
  \item voigt.py: contains functions to compute a Gaussian, Lorentzian, and 
        Voigt profile. Used by voigtcomp.py to compute the normalized 
        theoretical Voigt profile.
\end{itemize}

\subsection{results}
\label{sec:results}
The results directory contains the plots of the output of each test. It is 
similar in structure to code-output: each subdirectory is associated with 
the plots of the output of a particular RT/retrieval code, and these are numbered 
according to the time that the output was received. 

The plots subdirectory contains plots of the output of multiple codes for 
the comparison and retrieval tests. Subdirectories within the plots 
subdirectory are named according to the codes that are being compared and 
what is being compared (`comparison-iso' or `retrieval-inv'). For example, a 
comparison of `BART', `code', and `othercode' for the c01hjcleariso comparison 
test would be located at 
BARTTest/results/plots/BART{\_}code{\_}othercode{\_}c01hjcleariso/

\subsection{tests}
\label{sec:tests}
The tests/ directory contains the setups for each test. 
The subdirectory 00inputs/ contains the inputs used for tests. 
The remaining subdirectories are the tests; subdirectories named f{\#}{\#}name are forward 
model tests, c{\#}{\#}name are forward model comparison tests, s{\#}{\#}name are synthetic retrieval tests, and r{\#}{\#}name are real-data retrieval tests. 
Each subdirectory is detailed below. 
Note that paths in this section are given with respect to BARTTest/tests/.

The forward model tests are configured for Transit, the radiative transfer package of BART. 
The retrieval tests are configured for BART. 
These input and configuration files are provided so that the user may reference these for clarity on the exact setup for each test. 
The following sections describe each file in each subdirectory. 
Note that other RT or retrieval codes may handle inputs differently, such as in the case of collision-induced absorptions.

In general, each forward model test will have a Transit configuration file (e.g., broadening{\_}emission.trc), a Pylineread configuration file (e.g., broadening.plc), and an atmospheric model (e.g., broadening.atm). 
Some tests have multiple configuration files or atmospheric models, such as f05. 
For the synthetic retrieval tests, each directory will have two BART configuration files for transit and eclipse geometry (e.g., iso{\_}emission.brt), four text files containing the eclipse and transit depths and uncertainties (e.g., iso{\_}ecldepths.txt and iso{\_}ecluncs.txt), and a text file containing the paths to the filter files, with respect to the test directory.

\subsubsection{00inputs}
\label{sec:inputs}
00inputs contains the input files used for Transit (the radiative transfer package of BART) to execute the tests.

New (fake) molecules are defined to be used in the f01 -- f06 tests:
\begin{itemize} \itemsep0pt
  \item CG1, similar to H2
  \item CG2, similar to He
  \item CG3, similar to N2
  \item LG1, similar to H2O
  \item LG2, similar to CH4
  \item LG3, similar to CO
  \item LG4, similar to CO2
\end{itemize}
Note that due to the similarity between these molecules, the user may use 
the real molecule in place of the fake one as far as calculations are 
concerned (mass, isotopes, isotope ratios, etc.). However, it is important to 
note that the line lists for these fake molecules do not match the real line 
lists of the real molecules; be sure to use the supplied .par line lists.\\

Files in 00inputs/:
\begin{itemize} \itemsep0pt
  \item CIA{\_}HITRAN{\_}H2He{\_}0200-9900K{\_}0.5-500um.dat: An opacity table 
        for collision-induced absorptions between H2 and He molecules.
  \item hd189733b-fp00k2odfnew.pck: Kurucz stellar model file for HD 189733.
  \item HD189733b.tep: Transiting ExoPlanet file for HD 189733b. Contains 
        system parameters.
  \item litran.dat: A modified version of the HITRAN definitions file used by 
        the Pylineread module (reads line lists to build a file of line 
        transitions) of Transit. Defines the fake molecules listed above.
  \item molecules.dat: A modified version of the molecule definitions file 
        used by Transit. Defines the fake molecules listed above.
\end{itemize}

Subdirectories in 00inputs/:
\begin{itemize} \itemsep0pt
  \item filters/: contains filter files used for the retrieval tests.
  \item par/: contains line lists for the fake molecules, and files used to pull 
        HITRAN/HITEMP linelists from the HITRAN website.
  \item TLI/: contains the Transition Line Information files that Transit uses 
        as input. Note that this is initially empty.
\end{itemize}

\subsubsection{f01oneline}
\label{sec:oneline}
This test uses a model atmosphere that has one layer of 100\% LG1, and all 
remaining layers 85\% CG1 and 15\% CG2. LG1 has only one line at ~2.29 um. 
The test is executed in eclipse geometry.\\

Files in f01oneline/:
\begin{itemize} \itemsep0pt
  \item oneline.atm: Atmospheric model used by Transit.
  \item oneline.plc: Configuration file for pylineread. Creates line 
        list input file for Transit.
  \item oneline{\_}emission.trc: Configuration file for Transit. Produces 
        emission spectrum.
\end{itemize}

\subsubsection{f02fewline}
\label{sec:fewline}
This test uses a model atmosphere that has one layer of 100\% LG2, and all 
remaining layers 85\% CG1 and 15\% CG2. LG2 has three lines: ~2.15 um, 
~2.29 um, ~3.20 um. The test is executed in both eclipse and transit 
geometries.\\

Files in f02fewline/:
\begin{itemize} \itemsep0pt
  \item fewline.atm: Atmospheric model used by Transit.
  \item fewline.plc: Configuration file for pylineread. Creates line 
        list input file for Transit.
  \item fewline{\_}emission.trc: Configuration file for Transit. Produces 
        emission spectrum.
  \item fewline{\_}transmission.trc: Configuration file for Transit. Produces 
        transmission spectrum.
\end{itemize}

\subsubsection{f03multiline}
\label{sec:multiline}
This test uses a model atmosphere that has one layer of 100\% LG2, one layer 
of 100\% LG3, one layer of 100\% LG4, and all remaining layers 85\% CG1 and 
15\% CG2. LG2 has lines as in f02fewline; LG3 has lines at ~2.38 um, ~2.50 um, 
and ~2.54 um; LG4 has lines at ~2.86 um, ~3.02 um, and ~3.78 um. The test is 
executed in both eclipse and transit geometries.\\

Files in f03multiline/:
\begin{itemize} \itemsep0pt
  \item multiline.atm: Atmospheric model used by Transit.
  \item multiline.plc: Configuration file for pylineread. Creates line 
        list input file for Transit.
  \item multiline{\_}emission.trc: Configuration file for Transit. Produces 
        emission spectrum.
  \item multiline{\_}transmission.trc: Configuration file for Transit. Produces 
        transmission spectrum.
\end{itemize}

\subsubsection{f04broadening}
\label{sec:broadening}
This test looks at the accuracy of the Voigt profile shape for one line. 
One layer of the model atmosphere has 0.01\% LG1 and 99.99\% CG3, and 
all remaining layers 100\% CG3. The layer with LG1 is at a pressure of 
0.33516 bars and a temperature of 1442.58 K. The normalized theoretical Voigt 
profile is computed for this situation by calculating the half width at 
half max (HWHM) of a Gaussian due to thermal broadening and the HWHM of 
a Lorentzian due to pressure broadening. These are used as parameters for 
the Faddeeva function; the real part of this function is related to the 
Voigt profile. This is compared to the normalized opacity spectrum of the 
line as computed by Transit.\\

Files in f04broadening/:
\begin{itemize} \itemsep0pt
  \item broadening.atm: Atmospheric model used by Transit.
  \item broadening.plc: Configuration file for pylineread. Creates line 
        list input file for Transit.
  \item broadening{\_}emission.trc: Configuration file for Transit. Produces 
        emission spectrum.
\end{itemize}

\subsubsection{f05abundance}
\label{sec:abundance}
This test explores the effect of varying the abundance of a molecule in an 
optically-thin situation. If the atmosphere is optically thin and the line 
is weak, then doubling the abundance should result in a line that is twice as 
intense.

This test uses a uniform atmosphere composed of x\% LG1 and 100-x\% CG1 at 
all layers. X ranges from 0\% to 0.1\% in steps of 0.01\%. All 11 cases 
are executed in eclipse geometry.\\

Files in f05abundance/:
\begin{itemize} \itemsep0pt
  \item abundance{\_}line.plc: Configuration file for pylineread. Creates 
        line list input file for Transit.
  \item abundance{\_}*{\_}emission.trc: Configuration files for Transit. * 
        ranges from 0 to 1e-3 in steps of 1e-4.
  \item README: Brief readme about the test.
\end{itemize}

Subdirectories in f05abundance/:
\begin{itemize} \itemsep0pt
  \item atm/: Directory containing the atmospheric models. Naming convention 
        is, i.e. 1e-4{\_}uniform.atm, where 1e-4 corresponds to the abundance 
of LG1.
\end{itemize}

\subsubsection{f06blending}
\label{sec:blending}
This test uses two lines in close proximity to look at the 
result of two lines blending together. There is a LG1 line at 2.28919 um 
and a LG2 line at 2.28921 um. LG1 and LG2 are only present in 1 layer of 
the atmospheric model.\\

Files in f06blending/:
\begin{itemize} \itemsep0pt
  \item blending.atm: Atmospheric model used by Transit.
  \item blending.plc: Configuration file for pylineread. Creates 
        line list input file for Transit.
  \item blending{\_}emission.trc: Configuration file for Transit. Produces 
        emission spectrum.
\end{itemize}

\subsubsection{f07multicia}
\label{sec:multicia}
Transit incorporates collision-induced absorptions (CIAs) via a text file 
opacity table in a certain format. Other codes may also handle CIAs in a 
similar manner. This test uses an atmosphere uniformly composed of 85\% H\sb{2} 
and 15\% He at all altitudes. The forward model is generated for three cases: 
no CIAs, H\sb{2}-He CIAs, and H\sb{2}-H\sb{2} and H\sb{2}-He CIAs.\\

Files in f07multicia/:
\begin{itemize} \itemsep0pt
  \item multicia.atm: Atmospheric model used by Transit.
  \item multicia.plc: Configuration file for pylineread. Creates 
        line list input file for Transit.
  \item noCIA{\_}emission.trc: Configuration file for Transit. Produces 
        emission spectrum. No CIAs.
  \item oneCIA{\_}emission.trc: Configuration file for Transit. Produces 
        emission spectrum. H\sb{2}-He CIAs.
  \item twoCIA{\_}emission.trc: Configuration file for Transit. Produces 
        emission spectrum. H\sb{2}-H\sb{2} and H\sb{2}-He CIAs.
\end{itemize}

\subsubsection{f08isothermal}
\label{sec:isothermal}
This test checks the case of isothermal emission. For isothermal emission, 
the result should be a blackbody spectrum corresponding to the temperature 
of the atmosphere. Full line lists are used for multiple molecules 
Note that real molecules and line lists are used for this test, not the 
fake molecules and line lists used in previous tests.\\

Files in f08isothermal/:
\begin{itemize} \itemsep0pt
  \item isothermal.atm: Atmospheric model used by Transit.
  \item isothermal.plc: Configuration file for pylineread. Creates 
        line list input file for Transit.
  \item isothermal{\_}emission.trc: Configuration file for Transit. Produces 
        emission spectrum.
\end{itemize}

\subsubsection{f09energycons}
\label{sec:energycons}
This test runs forward models for a 60\% H\sb{2}, 10\% CO, 10\% CO\sb{2}, 10\% CH\sb{4}, and 10\% H\sb{2}O non-inverted atmosphere at different resolutions, integrates each spectrum, and compares the values. Since the spectra are not sampled at true line-by-line resolution, there should be minor (\lt 1\%) differences.

Files in f09energycons/:
\begin{itemize} \itemsep0pt
  \item energycons.atm: Atmospheric model used by Transit.
  \item energycons.plc: Configuration file for pylineread. Creates 
        line list input file for Transit. Note: exactly matches isothermal.plc.
  \item energycons{\_}*{\_}emission.trc: Configuration files for Transit. Produces 
        emission spectra. * indicates the number of steps made within a 1 cm\sp{-1} interval (default: 1, 2, 5, and 10 steps).
\end{itemize}

\subsubsection{c01hjcleariso}
\label{sec:comp-iso}
This test generates forward models for an HD 189733b-like planet with an isothermal PT profile in both eclipse and transit geometry. The atmosphere contains H, He, C, N, O, H\sb{2}, 
CO, CO\sb{2}, CH\sb{4}, H\sb{2}O, N\sb{2}, and NH\sb{3}. Line lists are 
included for H\sb{2}, CO, CO\sb{2}, CH\sb{4}, H\sb{2}O, and NH\sb{3}. CIAs 
are included for H\sb{2}-H\sb{2} and H\sb{2}-He.\\

Files in c01hjcleariso/:
\begin{itemize} \itemsep0pt
  \item comparison.plc: Configuration file for Pylineread. Creates 
        line list input file for Transit. Note: requires line lists. See 
        Section \ref{sec:commands} for information on downloading line lists if 
        the user wishes to download them. Be aware that this will take some 
        time and will use {\sim}15 GB of storage.
  \item iso.tea: Atmospheric model used by Transit.
  \item iso{\_}emission.trc: Configuration file for Transit. Produces 
        emission spectrum.
  \item iso{\_}transmission.trc: Configuration file for Transit. Produces 
        transmission spectrum.
\end{itemize}

\subsubsection{c02hjclearnoinv}
\label{sec:comp-noinv}
This test generates forward models for an HD 189733b-like planet with a non-inverted PT profile in both eclipse and transit geometry. The atmosphere contains H, He, C, N, O, H\sb{2}, 
CO, CO\sb{2}, CH\sb{4}, H\sb{2}O, N\sb{2}, and NH\sb{3}. Line lists are 
included for H\sb{2}, CO, CO\sb{2}, CH\sb{4}, H\sb{2}O, and NH\sb{3}. CIAs 
are included for H\sb{2}-H\sb{2} and H\sb{2}-He.\\

Files in c02hjclearnoinv/:
\begin{itemize} \itemsep0pt
  \item comparison.plc: Configuration file for Pylineread. Creates 
        line list input file for Transit. Note: requires line lists. See 
        Section \ref{sec:commands} for information on downloading line lists if 
        the user wishes to download them. Be aware that this will take some 
        time and will use {\sim}15 GB of storage.
  \item noinv.tea: Atmospheric model used by Transit.
  \item noinv{\_}emission.trc: Configuration file for Transit. Produces 
        emission spectrum.
  \item noinv{\_}transmission.trc: Configuration file for Transit. Produces 
        transmission spectrum.
\end{itemize}

\subsubsection{c03hjclearinv}
\label{sec:comp-inv}
This test generates forward models for an HD 189733b-like planet with an inverted PT profile in both eclipse and transit geometry. The atmosphere contains H, He, C, N, O, H\sb{2}, 
CO, CO\sb{2}, CH\sb{4}, H\sb{2}O, N\sb{2}, and NH\sb{3}. Line lists are 
included for H\sb{2}, CO, CO\sb{2}, CH\sb{4}, H\sb{2}O, and NH\sb{3}. CIAs 
are included for H\sb{2}-H\sb{2} and H\sb{2}-He.\\

Files in c03hjclearinv/:
\begin{itemize} \itemsep0pt
  \item comparison.plc: Configuration file for Pylineread. Creates 
        line list input file for Transit. Note: requires line lists. See 
        Section \ref{sec:commands} for information on downloading line lists if 
        the user wishes to download them. Be aware that this will take some 
        time and will use {\sim}15 GB of storage.
  \item inv.tea: Atmospheric model used by Transit.
  \item inv{\_}emission.trc: Configuration file for Transit. Produces 
        emission spectrum.
  \item inv{\_}transmission.trc: Configuration file for Transit. Produces 
        transmission spectrum.
\end{itemize}

\subsubsection{s01hjcleariso}
\label{sec:retrievals}
For this test, the isothermal emission and transmission spectra generated for 
f09comparison were binned according to 47 filters spanning 
1.97 -- 4.95 \microns (see BARTTest/tests/00inputs/filters/). 
Noise was added to the eclipse/transit 
depths by calculating the theoretical photon noise and pulling from a Gaussian 
centered on zero with a standard deviation equal to the photon noise. The test 
is to perform a retrieval of atmospheric parameters and abundances using this 
eclipse/transit depth data and compare the result to the known inputs.

Note that for the isothermal emission retrieval, the retrieved abundances will 
likely not match the known inputs. This is because, in principle, the 
abundances do not affect the resulting spectrum. Due to noise being added to 
an otherwise perfect blackbody, retrieval algorithms will try to 
fit the noise. The important result is that the retrieval PT profile is 
isothermal within 3$\sigma$, preferably within 1$\sigma$.\\

Files in s01hjcleariso/:
\begin{itemize} \itemsep0pt
  \item iso{\_}ecldepths.txt: Text file containing the eclipse depths 
        corresponding to the 47 filters.
  \item iso{\_}ecl{\_}noiseless.txt: Text file containing the noiseless eclipse depths.
  \item iso{\_}ecluncs.txt: Text file containing the eclipse depth 
        uncertainties corresponding to the 47 filters.
  \item iso{\_}emission.brt: BART configuration file to perform the retrieval. 
        Isothermal emission case.
  \item iso{\_}filters.txt: Text file containing the paths to the filter files corresponding to the depths.
  \item iso{\_}tradepths.txt: Text file containing the transit depths. Order 
        corresponds to the filter filters in 00inputs/filters/.
  \item iso{\_}tra{\_}noiseless.txt: Text file containing the noiseless transit depths.
  \item iso{\_}trauncs.txt: Text file containing the transit depth 
        uncertainties. 
  \item iso{\_}transmission.brt: BART configuration file to perform the 
        retrieval. Isothermal transmission case.
  \item retrievals.plc: Configuration file for Pylineread. Creatse line list input file for Transit.
\end{itemize}

\subsubsection{s02hjclearnoinv}
For this test, the noninverted emission and transmission spectra generated for 
f09comparison were binned according to 47 filters spanning 
1.97 -- 4.95 \microns (see BARTTest/tests/00inputs/filters/). 
Noise was added to the eclipse/transit 
depths by calculating the theoretical photon noise and pulling from a Gaussian 
centered on zero with a standard deviation equal to the photon noise. The test 
is to perform a retrieval of atmospheric parameters and abundances using this 
eclipse/transit depth data and compare the result to the known inputs.\\

Files in s02hjclearnoinv/:
\begin{itemize} \itemsep0pt
  \item noinv{\_}ecldepths.txt: Text file containing the eclipse depths 
        corresponding to the 47 filters.
  \item noinv{\_}ecl{\_}noiseless.txt: Text file containing the noiseless eclipse depths.
  \item noinv{\_}ecluncs.txt: Text file containing the eclipse depth 
        uncertainties corresponding to the 47 filters.
  \item noinv{\_}emission.brt: BART configuration file to perform the 
        retrieval. Noninverted emission case.
  \item noinv{\_}filters.txt: Text file containing the paths to the filter files corresponding to the depths.
  \item noinv{\_}tradepths.txt: Text file containing the transit depths. Order 
        corresponds to the filter filters in 00inputs/filters/.
  \item noinv{\_}tra{\_}noiseless.txt: Text file containing the noiseless transit depths.
  \item noinv{\_}trauncs.txt: Text file containing the transit depth 
        uncertainties. 
  \item noinv{\_}transmission.brt: BART configuration file to perform the 
        retrieval. Noninverted transmission case.
  \item retrievals.plc: Configuration file for Pylineread. Creatse line list input file for Transit.
\end{itemize}

\subsubsection{s03hjclearinv}
For this test, the inverted emission and transmission spectra generated for 
f09comparison were binned according to 47 filters spanning 
1.97 -- 4.95 \microns (see BARTTest/tests/00inputs/filters/). 
Noise was added to the eclipse/transit 
depths by calculating the theoretical photon noise and pulling from a Gaussian 
centered on zero with a standard deviation equal to the photon noise. The test 
is to perform a retrieval of atmospheric parameters and abundances using this 
eclipse/transit depth data and compare the result to the known inputs.\\

Files in s03hjclearinv/:
\begin{itemize} \itemsep0pt
  \item inv{\_}ecldepths.txt: Text file containing the eclipse depths 
        corresponding to the 47 filters.
  \item inv{\_}ecl{\_}noiseless.txt: Text file containing the noiseless eclipse depths.
  \item inv{\_}ecluncs.txt: Text file containing the eclipse depth 
        uncertainties corresponding to the 47 filters.
  \item inv{\_}emission.brt: BART configuration file to perform the retrieval. 
        Inverted emission case.
  \item inv{\_}filters.txt: Text file containing the paths to the filter files corresponding to the depths.
  \item inv{\_}tradepths.txt: Text file containing the transit depths. Order 
        corresponds to the filter filters in 00inputs/filters/.
  \item inv{\_}tra{\_}noiseless.txt: Text file containing the noiseless transit depths.
  \item inv{\_}trauncs.txt: Text file containing the transit depth 
        uncertainties. 
  \item inv{\_}transmission.brt: BART configuration file to perform the 
        retrieval. Inverted transmission case.
  \item retrievals.plc: Configuration file for Pylineread. Creatse line list input file for Transit.
\end{itemize}

\subsubsection{r01hd189733b}
HD 189733b is one of the most observed exoplanets, with observations across the NIR and MIR. 
It is therefore a good testcase for retrieval algorithms as the plethora of 
data points allows for the constraining of some parameters. 

This test uses space-based photometric and spectroscopic observations of the 
dayside emission of HD 189733b to perform a retrieval of atmospheric parameters 
and abundances. Data presented in the following publications are used:
\begin{itemize} \itemsep0pt
  \item Charbonneau et al. 2008, "The Broadband Infrared Emission Spectrum of the Exoplanet HD 189733b"
  \item Swain et al. 2009, "Molecular Signatures in the Near Infrared Dayside Spectrum of HD 189733b"
  \item Agol et al. 2010, "The Climate of HD 189733b from Fourteen Transit and Eclipses Measured by Spitzer"
  \item Knutson et al. 2012, "3.6 and 4.5 \micron Phase Curves and Evidence for Non-equilibrium Chemistry in the Atmosphere of Extrasolar Planet HD 189733b"
  \item Todrov et al. 2014, "Updated Spitzer Emission Spectroscopy of Bright Transiting Hot Jupiter HD 189733b"
\end{itemize}

Files in r01hd189733b/:
\begin{itemize} \itemsep0pt
  \item ecl_data.txt: Text file containing the eclipse data, with references for each.
  \item HD189733b.brt: Configuration file for BART.
  \item HD189733b.plc: Configuration file for Pylineread. Creates 
        line list input file for Transit. Note: requires line lists. See 
        Section \ref{sec:commands} for information on downloading line lists if 
        the user wishes to download them. Be aware that this will take some 
        time and will use {\sim}15 GB of storage.
\end{itemize}

\section{Example}
\label{sec:example}
This section will detail methods of running BARTTest. Since BARTTest is 
configured for Linux, all commands given are for that operating sytem.

First, the user must clone BARTTest from Github to their machine. 
Navigate in a terminal to the location where BARTTest will be cloned. 
Then type
\beginverbatim
git clone https://github.com/exosports/BARTTest BARTTest/
\endverbatim
to clone it into a directory called BARTTest/ within your current directory.
Then, navigate into the BARTTest directory.
\beginverbatim
cd BARTTest/
\endverbatim

BARTTest comes with BART's code output and results already there (except the large input files, like line lists and opacity files). For this example, we will start fresh as if that is not the case. Type
\beginverbatim
make clean
\endverbatim
into the terminal to delete all of the existing output from BART. Note that 
reproducing all of that output will take many, many CPU hours, so it will 
be easier to re-clone BARTTest than to run the code if the user wishes to 
un-do the deletion of that output.

From here, the execution method will be dependent on what test(s) the user 
wishes to execute. The common step to (almost) all of the options is
\beginverbatim
make bart
\endverbatim
as this command will clone and compile BART and its submodules from Github. 

To run the forward modeling cases that do not require line lists (f01 
-- f07), type
\beginverbatim
make quicktests
\endverbatim
into the terminal. It will execute f01 through f07 sequentially and produce 
plots of the output of those specific tests. Note that if the user deleted 
the existing BART output, there will be some error messages related to the 
files missing for other tests, like f08 and f09.

If the user wishes to run all of the forward modeling tests (f01 -- f09), full line lists are required.
Type
\beginverbatim
make hitran_linelists
\endverbatim
to download all of the necessary line lists. Then, type
\beginverbatim
make forwardtests
\endverbatim
into the terminal to run the tests. \textbf{Be aware of the hardware requirements for this} 
(see Section \ref{sec:requirements}). This will take on the order of hours 
to execute.

If the user wishes to run only the comparison tests (c01 -- c03), type
\beginverbatim
make comparisontests
\endverbatim
into the terminal. This will take roughly as long as the previous option, as 
the comparison test is the most time-consuming part of the forward modeling 
tests. Note that full line lists are required.

To run all of the synthetic retrievals, type
\beginverbatim
make synthretrievals
\endverbatim
into the terminal. \textbf{Be aware of the hardware requirements for this} 
(see Section \ref{sec:requirements}). This will take on the order of a week to complete.

If the user wishes to run, e.g., the synthetic retrievals on a non-inverted 
atmosphere in only eclipse geometry, type
\beginverbatim
make noinv_retrieval_e
\endverbatim

\section{How to Use This Software to Test RT and Retrieval Codes}
\label{sec:howto}
We would like you to contribute the results of your code to BARTTest! 

BARTTest has been designed with one main goal in mind: to provide verification 
that RT and retrieval codes work as intended. To do so, there are numerous 
tests that are offered in the package (see Section \ref{sec:tests}).

It is not absolutely necessary for all tests to be executed by some RT code. 
Rather, the comparison tests (see Sections \ref{sec:comp-iso} -- \ref{sec:comp-inv}) are the most important 
of the forward modeling tests due to the realistic setup; this best reflects 
the real cases RT codes are used for. Similarly, the HD 189733b case is the most 
important of the retrieval tests since it is based on real data. The more 
codes that yield the same (or similar) answer, the more confident we as a 
community can be that our codes are all implementing the same mathematics, 
and thus the more confident we can be in the results of our codes. If there 
are discrepancies between the results of different codes, the simpler forward 
modeling tests and synthetic retrieval tests may be able to diagnose the origin 
of those differences.

In order to contribute results, the user must run their code for the exact 
same cases as BARTTest. The user may provide a file 
(in BARTTest/code-output/{\textless}yourcode{\textgreater}/) to re-format 
their code output to that of BART's output so that existing plotting functions 
may be used to produce plots of the results. Rather than attempting 
to explain this format, the user is encouraged to look at the output of BART 
in BARTTest/code-output/01BART/. If the user does not wish to reformat the 
output of their code to match this particular format, then they must include a 
file (in BARTTest/code-output/{\textless}yourcode{\textgreater}/) to produce 
plots of the results of their code 
(in BARTTest/results/{\textless}yourcode{\textgreater}/). 
This is to ensure that the plots provided can be easily reproduced.

Contributing results to BARTTest will follow a particular format. The 
code-output/ and results/ directories will have subdirectories for each RT 
code following a specific naming convention of {\#}{\#}code/ (e.g., 01BART/) as 
described in Section \ref{sec:output}. The user may name the directory for 
their code output/results as they wish. The number 
preceeding the name of the code is determined by the order 
in which results are received via a pull request on Github; the user may pick 
a number but it may be changed depending on the order it is received (e.g., if 
two users submit results around the same time). If the user does not pick 
a brief name for their code, the authors will pick one as they see fit, 
whether based on the name of the code (e.g., BART) or based on the name of 
the author(s) if no name exists. \textbf{Note that we do not ask the user 
to submit their RT/retrieval code itself! We only want the output of the code, as well as any other 
necessary files such as code used to convert the output to BART's format or 
code to produce plots of the output if it has not been re-formatted to 
BART's format.} Users that convert the output of their code to the same format 
as BART can use the existing plotting routines to produce plots of the 
results if they wish to save the maintainers of BARTTest some time. Note that 
if the re-formatted output is incorrect, the maintainers of BARTTest will not 
fix that and will leave the pull request open until the user fixes it.

\end{document}
